\documentclass[11pt]{beamer}
\usetheme{default}

\usepackage[utf8]{inputenc}
\usepackage[spanish]{babel}
\usepackage{amsmath}
\usepackage{amsfonts}
\usepackage{amssymb}
\usepackage{graphicx}

\author{Nicolás Ripoll} %\\ Estudiante de Dr. en Ing. Mecánica}
\title{IMP462 \\ Fundamentos de la combustión turbulenta}

\subtitle{Parte 1: Derivación de ecuaciones de conservación}

%logo{}

\institute{Departamento de Ingeniería Mecánica \\Universidad Técnica Federico Santa María}

\date{\\ martes 10 de enero, 2016}

\subject{Fundamentos de la combustion turbulenta}

\setbeamercovered{transparent}

\setbeamertemplate{navigation symbols}{}

\begin{document}
	\maketitle
	
%	Diapositiva 1
	\begin{frame}
		\frametitle{¿Qué es la combustión?}
	\end{frame}

%	Diapositiva 2
	\begin{frame}
		\frametitle{Análisis del Volumen de Control}
	\end{frame}
	
	\begin{frame}
		\frametitle{Formas Generales}
	\end{frame}
	
	\begin{frame}
		\frametitle{Conservación de Momento}
	
	
		\begin{equation} % Separación entre tensor de esfuerzos viscosos y presión
			\dfrac{\partial}{\partial t} \left(\rho u_{j}\right) + \dfrac{\partial}{\partial x_{i}} \left(\rho u_{i} u_{j}\right) = -\dfrac{\partial p}{\partial x_{j}} + \dfrac{\partial \tau_{i j}}{\partial x_{i}} + \rho \sum_{k=1}^{N} Y_{k} f_{k,j}
		\end{equation}

		Hola

		\begin{equation} % tensor de esfuerzos total
			\dfrac{\partial}{\partial t} \left(\rho u_{j}\right) + \dfrac{\partial}{\partial x_{i}} \left(\rho u_{i} u_{j}\right) = \dfrac{\partial \sigma_{i j}}{\partial x_{i}} + \rho \sum_{k=1}^{N} Y_{k} f_{k,j}
		\end{equation}
		
	\end{frame}
	
\bibliographystyle{plain}
\bibliography{biblio.bib}
\end{document}